\documentclass{sig-alternate}

\usepackage{xspace}
\newcommand\Motmot{\textsc{Motmot}\xspace}

\begin{document}

\CopyrightYear{2012}

\title{Motmot: A Distributed Chat Protocol}

\numberofauthors{5}
%
\author{
  \alignauthor
  Carl Jackson \\
    \affaddr{Harvard University} \\
    \email{cjackson@college.harvard.edu}
  %
  \alignauthor
  Max Wang \\
    \affaddr{Harvard University} \\
    \email{max.wang@college.harvard.edu}
  %
  \alignauthor
  EJ Bensing \\
    \affaddr{Harvard University} \\
    \email{ebensing@college.harvard.edu}
  %
  \and
  \alignauthor
  Jeff Atwood \\
    \affaddr{Harvard University} \\
    \email{jatwood@college.harvard.edu}
  %
  \alignauthor
  Julie Hansbrough \\
    \affaddr{Harvard University} \\
    \email{jhansbrough@college.harvard.edu}
}

\maketitle

\begin{abstract}

\Motmot is a new distributed text chat protocol based on Paxos. Unlike previous
chat protocols like XMPP, OSCAR, and IRC, \Motmot provides guaranteed message
delivery, total message ordering, end-to-end encryption, peer authentication,
and message deniability.
% TODO: write more

\end{abstract}

\section{Introduction}

\section{libmotmot}

Guaranteed delivery and total message ordering are perhaps the most fundamental
chat features provided by \Motmot.  In order to obtain these properties, the
clients participating in a chat must reach a consensus concerning the reception
and order of chat events.  \Motmot assumes a benign failure model and uses a
variant of the Paxos consensus protocol \cite{paxos}, which is the core of its
client library \texttt{libmotmot}, in order to provide these guarantees.

\subsection{Design}



\subsection{Implementation}

\subsection{Testing}

\subsection{Lessons Learned}

\section{Discovery Server}

\section{Pidgin Plugin}

To allow our protocol to be utilized via Pidgin, we needed our plugin to interface with two major aspects of the protocol: the discovery server, and the libmotmot library.

\subsection{Plugin Interactions with Discovery Server}

\subsection{Plugin Interactions with Libmotmot}
There were a few major difficulties that had to be addressed when calling libmotmot within the plugin:

\begin{itemize}
\item libpurple makes a distinction between "IMs" (two-person chats) and "chats" (three or more people); libmotmot does not.  libpurple takes in different arguments for its IM-related and its chat-related functions

\item One cannot modify the main Pidgin event loop from within a plugin (without patching libpurple itself).  Instead, callbacks are made to plugin functions from within the main event loop, so finding a way to call libmotmot functions correctly (i.e. still honoring the correct control flow from libmotmot's perspective) and consistently within the bounds of these plugin functions was a second challenge.

\end{itemize}

\section{Conclusions}

Wheeee.

\section{Acknowledgments}

\thebibliography{2}{
  \bibitem{paxlive}
    T. Chandra, R. Griesemer, and J. Redstone.
    Paxos made live---an engineering perspective.
    In \emph{Proc. 26th Symp. on Principles of Distributed Computing}, pp. 398-407, Aug. 2007.

  \bibitem{paxos}
    L. Lamport.
    The part-time parliament.
    \emph{ACM Transactions on Computer Systems}, 16(2):133-169, 1998.

  \bibitem{paxsimp}
    L. Lamport.
    Paxos made simple.
    \emph{ACM SIGACT News 32}, 4:18-25, Dec. 2001.

  \bibitem{rfc}
    P. Resnick, ed.
    Internet Message Format, RFC 2822.
    April 2001.
}

\end{document}
