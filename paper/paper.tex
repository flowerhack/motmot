\documentclass{sig-alternate}

\usepackage{xspace}
\newcommand\Motmot{\textsc{Motmot}\xspace}

\begin{document}

\CopyrightYear{2012}

\title{Motmot: A Distributed Chat Protocol}

\numberofauthors{5}
%
\author{
  \alignauthor
  Carl Jackson \\
    \affaddr{Harvard University} \\
    \email{cjackson@college.harvard.edu}
  %
  \alignauthor
  Max Wang \\
    \affaddr{Harvard University} \\
    \email{max.wang@college.harvard.edu}
  %
  \alignauthor
  EJ Bensing \\
    \affaddr{Harvard University} \\
    \email{ebensing@college.harvard.edu}
  %
  \and
  \alignauthor
  Jeff Atwood \\
    \affaddr{Harvard University} \\
    \email{jatwood@college.harvard.edu}
  %
  \alignauthor
  Julie Hansbrough \\
    \affaddr{Harvard University} \\
    \email{jhansbrough@college.harvard.edu}
}

\maketitle

\begin{abstract}

\Motmot is a new distributed text chat protocol based on Paxos. Unlike previous
chat protocols like XMPP, OSCAR, and IRC, \Motmot provides guaranteed message
delivery, total message ordering, end-to-end encryption, peer authentication,
and message deniability.
% TODO: write more

\end{abstract}

\section{Introduction}

\section{libmotmot}

\subsection{Design}

\subsection{Implementation}

\subsection{Testing}

\subsection{Lessons Learned}

\section{Discovery Server}

\section{Pidgin Plugin}

\section{Conclusions}

Wheeee.

\section{Acknowledgments}

\thebibliography{2}{
  \bibitem{paxos}
    L. Lamport.
    The part-time parliament.
    \emph{ACM Transactions on Computer Systems}, 16(2):133-169, 1998.
}

\end{document}
