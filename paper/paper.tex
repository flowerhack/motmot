\documentclass{sig-alternate}

\usepackage{enumitem,url,xspace}
\newcommand\Motmot{\textsc{Motmot}\xspace}
\newcommand\libmotmot{\texttt{libmotmot}\xspace}
\newcommand\libpurple{\texttt{libpurple}\xspace}

\setdescription{leftmargin=0pt}

\begin{document}

\CopyrightYear{2012}

\title{Motmot: A Distributed Chat Protocol}

\numberofauthors{5}
%
\author{
  \alignauthor
  Carl Jackson \\
    \affaddr{Harvard University} \\
    \email{cjackson@college.harvard.edu}
  %
  \alignauthor
  Max Wang \\
    \affaddr{Harvard University} \\
    \email{max.wang@college.harvard.edu}
  %
  \alignauthor
  EJ Bensing \\
    \affaddr{Harvard University} \\
    \email{ebensing@college.harvard.edu}
  %
  \and
  \alignauthor
  Jeff Atwood \\
    \affaddr{Harvard University} \\
    \email{jatwood@college.harvard.edu}
  %
  \alignauthor
  Julie Hansbrough \\
    \affaddr{Harvard University} \\
    \email{jhansbrough@college.harvard.edu}
}

\maketitle

\begin{abstract}

\Motmot is a new distributed text chat protocol based on Paxos. Unlike previous
chat protocols like XMPP, OSCAR, and IRC, \Motmot provides guaranteed message
delivery, total message ordering, end-to-end encryption, peer authentication,
and message deniability.
% TODO: server federates, etc.
% TODO: no xml! It's a feature
% TODO: write more

\end{abstract}

\section{Introduction}

We wrote \Motmot due to our frustrations with existing chat protocols. XMPP, the
protocol underlying Google Chat, is a federated XML-based chat protocol. Most
annoyingly, we found that XMPP did not guarantee message delivery---when one
of the users' connection dies, messages sent around the time of the
disconnection may be lost, without any indication in the protocol that this has
happened. Furthermore, XMPP messages are not totally ordered, and different chat
participants can have different chat logs based on when messages were received.

Finally, XMPP was an unsatisfying protocol technologically. While the protocol
itself is federated, all chat messages are still sent to central servers for
relaying. In addition to the obvious single-point-of-failure concerns, there are
also several product-level concerns. In particular, any server which relays your
messages is able to read the entire contents of the chat. Furthermore, your only
guarantee that you are in fact talking to who you think you are comes from the
servers you are talking to.

Other chat protocols besides XMPP share similar properties. Skype, another
popular chat protocol, acts in a peer-to-peer manner.\footnote{Recent rumors
claim that Skype, under Microsoft's kindly direction, has moved some of the
formerly distributed parts of the Skype protocol to a centrally managed server
cluster. In light of such developments, it's unclear just how peer-to-peer
Skype remains.} However, the protocol (and its primary implementation) are
closed-source, so its guarantees cannot be further examined. As firm believers
in sunlight and open protocols, we found Skype also wanting.

\subsection{Design Goals}

We set out implementing \Motmot with several design goals in mind.

\begin{description}
    \item[Multi-party] Chat sessions can occur between any number of
    individuals.
    \item[Reliability] All chat messages should be reliably delivered. Any message
    recorded in the chat history is guaranteed to be delivered to the rest of the
    \Motmot session.
    \item[Total Ordering] All chat participants will receive the same messages
    in exactly the same order.
    \item[Security] All connections in the system will be encrypted with TLS,
    using certificates (and client certificates) to authenticate the other party
    to the communication.
    \item[Deniability] While participants in the chat are confident that the
    messages they receive are sent by the individuals specified, chat
    participants cannot prove the provenance of any message to any third party.
    In other words, after the chat has completed, all participants can fabricate
    any equivalently valid chat history.
    \item[Federation] Anyone can host their own server and issue authentication
    credentials. Any \Motmot user can talk to any other \Motmot user (with an
    identity in any realm) who correctly implements the protocol.
    \item[Presence] Users can maintain buddy lists, and know the connection
    status (online, busy, idle, etc.) of any buddy.
\end{description}

\section{libmotmot}

Guaranteed delivery and total message ordering are perhaps the most fundamental
chat features provided by \Motmot.  In order to obtain these properties, the
clients participating in a chat must reach a consensus concerning the reception
and order of chat events.  \Motmot assumes a benign failure model and uses a
variant of the Paxos consensus protocol \cite{paxos,paxsimp}, which is the core
of its client library \libmotmot, in order to provide these guarantees.

\subsection{Simplification}

\libmotmot implements a slightly simplified variant of the Multi-Paxos
algorithm, augmented by a number of out-of-band yet nonetheless indispensable
side protocols.  This core Paxos implementation is wrapped by a client
interface and uses MessagePack \cite{msgpack} for network communication.
As with the other components of \Motmot, \libmotmot makes use of the GNOME
Library (GLib) \cite{glib}.

The Paxos implementation of \libmotmot makes a fundamental simplification to
the standard algorithm: we do not allow participants who leave the protocol, by
failure or otherwise, to rejoin.  In the original Paxos protocol, protocol
participants may fail benignly or depart the protocol at any time, and return
at some later point to continue voting and learning values in the system.
This is necessary in order for participants to maintain a notion of persistent
identity.  It is also needed to ensure recovery in the event of total system
failure.  Participants keep a persistent log of the values committed in the
protocol.  In the event of total failure, they can individually come back up,
rejoin the protocol, and corroborate on their individual copies of the log,
thereby allowing the protocol to continue where it left off.

A chat protocol, however, has no need for this mode of recovery, nor for this
particular notion of identity.  The persistent identity which users want from
a chat service can be maintained out-of-band.  If a user leaves a chat, there
is no need, at the level of the consensus protocol, to treat the user as the
\emph{same} user if they returns to the chat later.  Users need not be informed
of the messages that were exchanged in their absence.\footnote{We have not yet
investigated whether or not a distributed chat protocol is able to support
offline messages cleanly.}  So long as the system is able to determine, in a
consistent manner, that the user has left, the user can gain a new identity
within the protocol upon his or her return.

Similarly, if a chat session ends, either due to failure or simply because all
users leave, the termination is allowed to be final.  The participants in the
chat may start a new session if they wish, but this new session need not have
any memory nor knowledge of the old one.

This simplification to Paxos yields many benefits.  For instance, clients may
still choose to keep a record of chats in a local file, but by disallowing
protocol rejoins, this persistent storage does not play any role in the safety
or progress of \libmotmot's Paxos implementation.  We can also avoid the need
for complicated and high-volume recovery mechanisms.  In \libmotmot's Paxos,
as in the final protocol presented in the original Paxos paper, only one
participant is designated as the proposer, and hence allowed to issue prepares
and subsequently decrees, at any given time.  In the absence of rejoins, we
are able to provide a very useful system invariant which full Paxos cannot
have:
\begin{quote}
  If a proposer successfully completes a prepare phase, that proposer will
  remain the protocol's designated proposer until it leaves the protocol, by
  failure or otherwise.
\end{quote}
We discuss this invariant further in the following subsection.

\subsection{Support Protocols}

In order to make Paxos efficient and usable, the core protocol needed to be
supplemented with additional protocols which, despite being out-of-band, were
no less integral to the functionality of \Motmot.  We will describe each of
these side protocols in detail.

\begin{description}
  \item[Join, part, and election:]
    The original Paxos paper mentions, but does not actually provide, a
    procedure for electing a Paxos system's designated proposer; nor does it
    describe mechanisms for adding and removing individuals from the system.
    These are all, however, necessary to the functionality of our consensus
    algorithm.  Election is necessary in order to ensure that the system makes
    progress past the prepare phase of Paxos, lest multiple participants
    attempt repeatedly to prepare over one another.  Join and part are required
    in order to support any usable notion of chat.

    In \libmotmot, we guarantee the consistency of joins, parts, and election
    results by baking them in to the standard commit protocol.  Each protocol
    participant keeps a list of all the participants in the chat, ordered
    by session-unique identifiers.  Participants can be added and removed from
    this list only through Paxos commits decreeing either \verb`DEC_JOIN` or
    \verb`DEC_PART`, which ensures the consistency of the participants in this
    list and their order.  Participants are identified by the ID of the Multi-%
    Paxos instance which decreed their join.

    Proposer election is performed locally, with each participant simply
    recognizing the earliest-joined participant to whom they have a live socket
    connection as the designated proposer.  Participants will only accept a
    prepare phase issued by this designated proposer.  When a new proposer
    successfully completes its prepare phase, it broadcasts a decree requesting
    that all participants evict the old proposer from their participant lists.
    In this manner, we are able to guarantee the invariant described above.

    This mechanism may occasionally result in the forced eviction of a live
    participant who was the previous proposer.  It is conceivable that transient
    network problems might cause a majority of participants to lose their
    connection to the original proposer, and hence to accept the prepare of
    the new proposer.  Before the new proposer is able to issue its decree
    parting the old proposer, however, these connections may all be
    reestablished, and the old proposer may be booted even though the system
    as a whole currently recognizes its presence.  This causes no loss of
    consistency and merely forces an old proposer in this situation to rejoin
    the chat.  We consider this a worthwhile tradeoff given the usefulness
    of our invariant.

  \item[Greet protocol:]
    The presence of joins and parts demands a mechanism for initiating
    communication with new protocol participants.  When a join is successfully
    accepted by a majority, the proposer will decree it.  Subsequently, the
    proposer will send the new participant a welcome message, which transfers
    all of its protocol state to the new participant.  The proposer always has
    the most up-to-date state, and since the network guarantees FIFO message
    ordering and the proposer is the only one who can issue commands to modify
    protocol state, the new participant's state will be consistent.  The
    proposer also assigns the new participant its ID (i.e., the instance
    number of its join decree).

    The new participant will then open sockets to each participant on the
    list it receives from the proposer.  It will then send each participant
    a hello message.  This hello message is also used in the reconnect protocol
    and is used to bind sockets to the protocol state objects representing
    participants.

  \item[Out-of-band requests:]
    The proposer is the only protocol participant able to order and to decree
    a chat message.  Hence, all participants must have a mechanism for asking
    the proposer to decree a message.  These requests also apply for joins
    (though not to parts, since we do not support forced kicking of users from
    a chat).

    We want to avoid stressing the network, and in particular want to keep the
    volume of data the proposer must handle to a minimum, since the proposer is
    one of the system's primary bottlenecks.  Thus, we did not want to force
    the proposer to broadcast the arbitrarily-large chat messages produced by
    the various users in the chat.  Instead of sending along the actual message,
    the requester of a message generates a session-unique identifier for the
    message, consisting of the requester's own participant ID and a locally
    unique request ID.  The requester then takes on the responsibility of
    broadcasting this identifier along with the actual message contents to all
    participants.  When this message is received, each participant caches it
    locally.  The proposer will make a decree which contains the identifier of
    the request; when the other participants receive a commit for the decree,
    they can do a local lookup for the message contents.

    This has the added benefit of providing all participants an assurance that
    message contents come directly from their sources without the possibility
    of corruption by the proposer.  At worst, the proposer can modify the
    cache index of the message.

    Because our request mechanism is done out-of-band, commits for a given
    request may arrive after the message contents are received from the initial
    broadcast.  We implement a basic protocol for retrieving and resending
    these messages if they cannot be found.

  \item[Reconnection:]

  \item[In-order learn:]
    \libmotmot's clients are notified of Paxos commits via a set of callbacks.
    In order to ensure the total ordering of chats as seen by the user, we
    ``learn'' about commits in order of Paxos instance number.  This means that
    a missing commit message could prevent client-side progress.  To rectify
    this, we implement a straightforward means for any participant to ask the
    proposer to remind them of a committed value.

  \item[Log synchronization and truncation:]
    Commits made in Paxos must be kept by all protocol participants in order
    to obtain Paxos's correctness guarantees.  Since we keep all of our logs
    in memory, however, we need a mechanism for pruning information which
    the entire system has finished learning.

    We implement an out-of-band sync-and-truncate protocol, which is used by
    the proposer to issue consistent system-wide deletion of old commits.  A
    proposer asks all protocol participants to return the last contiguous
    commit which their respective clients have learned about.  Once it receives
    a response from \emph{all} participants, it can take the minimum instance
    number of those commits and safely command all participants to truncate
    up to that point.

  \item[Connection continuations:]
\end{description}

\subsection{Implementation}

\subsection{Testing}

Testing \libmotmot in a realistic setting was also challenging. We resorted to
client ``fuzzing'' as our primary testing strategy, in which we randomly issued
valid commands to a pool of connected \Motmot instances. These commands included
messages, join invitations, and parts, and were issued at random intervals to
random chat participants.

One of the biggest challenges was writing a testing framework that allowed us to
script interactions between many processes. Our first attempt was a Ruby DSL
that allowed small scripts to spawn and interact with a given number of \Motmot
instances. After performing arbitrary manipulations to the client pool,
including sending messages, creating new clients, and killing clients, a reaper
thread ensures message ordering.

The second iteration of a testing framework instead relied more heavily on UNIX
abstractions. First, we create a UNIX named pipe to act as \Motmot's input
channel (replacing \texttt{stdin}). A small utility program
(\texttt{test/motmot}) then assists in \texttt{dup2}-ing the correct file
descriptors in place and spawning the correct process. This allows many
\Motmot's to be spawned in separate windows, while centralizing the control of
the instances. Unlike the previous testing framework, \texttt{stdout} is not
checked. Finally, a central trace runner writes commands to the appropriate
named pipes, causing messages to be sent to various windows.

A key debugging feature of both of these frameworks was the ability to attach
debuggers to every \Motmot instance spawned. This served to be a critical
debugging tool, especially when analyzing segmentation faults or ensuring client
state consistency.

One final problem when testing \libmotmot was in simulating the network.  Chats
occur over TCP in the real world, which have some amount of latency and can
perform packet fragmentation, neither of which is a concern when interfacing
over UNIX domain sockets. To simulate a lossy network (on which connections
occasionally die), we wrote a network proxy, which ferried data from one UNIX
socket to another, fragmenting them and adding between 5 and 500 milliseconds of
latency. Connections were also occasionally dropped at some random interval.

While simulating a network proved difficult, we found that fuzzing with network
delays was a very effective way of finding bugs. After sending tens of millions
of messages through \libmotmot, we are fairly confident that it correctly
implements Paxos and delivers chats reliably, even at very high message
throughputs (on the order of a thousand messages a second on low-latency
networks).

\subsection{Lessons Learned}

Our simplification to Paxos is a lesson in the importance of end-to-end design
\cite{end2end}.  The \libmotmot library is a backend protocol layer, on top of
which chat semantics are implemented.  Without taking the semantics of the
system's endpoints---the chat clients---into consideration, our consensus
algorithm would have been unnecessarily featureful and complicated.

The importance of out-of-band effort in distributed protocols was also made
very clear to us in our implementation.  Other implementors of Paxos have also
discovered the substantial amount of support needed to make the the protocol
work in practical production-level settings \cite{paxlive}.  Clean designs
and interfaces for these additional protocols was crucial in keeping our
system simple, functional, and understandable.

\section{Discovery Server}

The discovery server performs several tasks in the system: authentication and
identity federation; presence management; and connection brokering.

\subsection{Identity}

Much like other federated protocols like XMPP and email, identity handles are
based upon DNS. Every domain name with a SRV record can act as a \Motmot
identity realm, and give out email-like addresses (an \verb`addr-spec` as
defined in RFC 2822) for \Motmot communication.

To grant a user an identity on a realm, the user first presents a username and
password to the server, after which the server delegates authority for that name
to the client by signing that client's key. This key-signing process is the
method by which peers can authenticate securely to each other without a
centralized broker.

\subsection{Presence}

In chat protocols, presence is the system by which users can specify a list of
buddies to publish their online status to. The \Motmot discovery server provides
a presence lookup service, by recording the status of all of the clients
connected to it. Servers then push presence updates to all of the other servers
on any user's buddy list, which in turn push the notification to the relevant
clients.

This subsystem system does not need to be robust or reliable, since all pushed
state can be recreated from other sources, and since asynchronously pushed
presence notifications do not need to be perfectly reliable to ensure a usable
user interface.

\subsection{Brokering}

Finally, the server must perform message brokering. For clients behind firewalls
or routers which perform network address translation (NAT), direct connections
are not possible. In such a case, the discovery server might be forced to assist
in connecting the clients.

Technologies like STUN, TURN, and ICE provide implementations of ``NAT poking,''
and while mechanisms for interfacing with such servers were considered in the
design of the server, performing brokering itself was outside the scope of this
version of the protocol.

\subsection{Implementation}

The \Motmot discovery server was implemented in Python using several open source
libraries. The server itself is a custom RPC-like reactor/dispatcher written
using GEvent \cite{glib}, a coroutine-based networking library. For our
communication protocol, we use MessagePack \cite{msgpack} to serialize messages.
This allows for easy cross-language communication with clients and other servers,
since implementations are available for most popular languages.

We found that these two libraries provided good abstractions for the task of
implementing a server. Gevent's coroutines provided an extremely useful
concurrency model: they were executed in a single thread, so synchronization
around global objects was not necessary, while avoiding the complexity (and
unnecessary indentation) of a callback-based system. MessagePack also provided
an excellent abstraction, allowing the server to assume that arbitrary data
structures could be written to and read from the wire. The ad-hoc nature of the
format also helped speed development of the server and protocol.

% TODO: cite pyopenssl
Finally, we used \texttt{openssl} (and the \texttt{pyOpenSSL} library in
particular) to perform certificate operations.

\subsection{Lessons Learned}

We rewrote the server several times because we did not understand a few of the technical requirements of our design choices; namely, that communication needs to be asynchronous. Asynchrony allows for  push notifications to go over the same socket as all other requests. Once we understood that communication needed to by asynchronous, the next  design choice was how to implement socket reading. In an earlier implementation, we were inadvertently throwing away parts of messages, causing incorrect behavior. Moving to the unpacker object in the msgpack library and using its feed() function solved this problem and prevented data loss.

From there, the implementation was fairly straight forward. Servers currently authenticate by checking that the connection is coming from the IP that DNS records have for the domain. This could be made more robust (and is perhaps a concern for future versions), but as of the present, it allows for servers to be easily added to the global network and communicate.

Multiplexing over a single socket was by far the most difficult aspect of this server. Many of the problems with the original synchronous implementation could have been solved by adding a second socket, but we decided that using a single socket is superior because it has simpler failure modes.

We quickly learned that writing good test code for distributed systems is both difficult and time consuming. The server test code, while not exhaustive, covers many common situations. However, testing for partial failure of messages and other network related failures will be implemented in a future version.   Additionally, in a future version, we plan to add message IDs to each call so that the client can know which response corresponds to which request.

% TODO: EJ?

\section{Pidgin Plugin}

To allow our protocol to be utilized via Pidgin, we needed our plugin to
interface with two major aspects of the protocol: the discovery server, and the
\libmotmot library.  A \libpurple protocol plugin is compiled as a statically
linked library and installed into the lib directory of Pidgin. Upon start-up
Pidgin loads each of the protocol plugins in its protocol directory. If the
protocol is successfully loaded, the user can then create accounts of that
specific protocol.  In order to implement a protocol, the programmer must
implement a certain number of functions specifying the code to run when the user
takes an action, for example logging-in or joining a chat, in addition to
listening for input over a socket. Implemented functions are stored in a struct.
Calls to \libmotmot and message passing between the client and the discovery
server had to be implemented within these functions.

\subsection{Plugin Interactions with Discovery Server}
The procedure for hooking up the plugin to the discovery server is similar to
the procedures in other chat protocols. Two user-info fields were added to the
account creation menu, giving the server and port of the discovery server. In
the login code, a connection with the discovery server is initiated, and once
the connection is initiated an input listener is added and authentication
credentials are sent. Once authentication succeeds, status updates and friend
requests are handled. We used message pack in order to have a standard message
format across multiple platforms.

The input listener listens for input from the discovery server. Upon receiving a
message it deserializes the message from the message pack buffer, finds the
opcode, and calls the appropriate \libpurple functions based on the opcode and
the rest of the data sent by the message. Writes to the discovery server are
handled in individual purple functions. Based on the user action, a message pack
object is created, and the data is sent to the server across the connection.
Messages are asynchronous. Status updates also include the address and port
of a user so the client can know the location of a user to connect to.

The code to interact with the discovery server currently has a bug involving
status updates after accepting a friend request.

\subsection{Plugin Interactions with \libmotmot}
There were a few major difficulties that had to be addressed when calling
\libmotmot within the plugin:

\begin{itemize}
\item \libpurple makes a distinction between ``chats'' and ``IMs''; \libmotmot
does not.  (In particular, IMs are defined as a direct two-person communication
only, whereas chats are an overarching category that can involve one, two, or
more users.) \libpurple takes in different arguments for its IM-related and its
chat-related functions

\item The main Pidgin event loop cannot be modified from within a plugin
(without patching \libpurple itself).  Instead, callbacks are made to plugin
functions from within the main event loop, so we had to find a way to call
\libmotmot functions correctly (i.e. still honoring the correct control flow
from \libmotmot's perspective) and consistently within the bounds of these
plugin functions

\item \libmotmot and \libpurple use different handles to differentiate between
chats and users, and translating between the two handles proved challenging,
especially given the complexity of the information associated with \libpurple
chats (e.g. a single chat may have an id, a room name, \verb`PurpleConnection`
and \verb`PurpleConvChat` objects...)

\end{itemize}

The bulk of the integration work was writing callbacks for the
\verb`motmot_init` function of \libmotmot.  Once these callbacks are written
correctly, \libmotmot handles most of the work of starting, sending, and
coordinating chats automatically.  These callbacks were, in almost all
instances, responsible for calling \libpurple functions, which handle the UI
aspect of the associated \libmotmot calls.  In order to pass relevant
information to our callbacks, we created a \verb`MotmotInfo` struct, a large
data struct that is used differently in different callbacks to get all the
necessary info to the \libpurple calls. \verb`MotmotInfo` contained the info
necessary to mediate between \libpurple and \libmotmot. It contained both the
id of the chat which was necessary for \libpurple and the opaque data handle
necessary for \libmotmot.

We faced a challenge when writing the \verb`enter` function.  This function is a
callback responsible for allowing a user to enter a chat.  Due to the structure
of \libmotmot, the only information passed into \verb`enter` as an argument is
an internal handle, and we need create some sort of external handle.  We solved
this by creating a global variable, \verb`GLOBAL_ACCOUNT`, which represents the
account that a user first logs in under, and we associate our external handle
with this \verb`GLOBAL_ACCOUNT`.  Though this ``works'' if a user only signs in
with one account, this breaks when a user decides to sign in with multiple
accounts at once (something that \libpurple allows), and should be addressed in
future versions.

We were unable to fully integrate the \libmotmot protocol in the plugin.
The code is outlined with in the plugin file, however it is not currently
functional. Hopefully this will be fixed in future versions so that
\libmotmot can be used within a \libpurple plugin.

\section{Conclusions}

\Motmot is a novel distributed chat protocol that offers features not present in
other chat protocols. Its reliance on the Paxos algorithm allow it to react
properly to the changing topology of chatting peers, and is the core improvement
on other protocols.

While we have not had a chance to test the protocol in practice, we believe that
\Motmot not only provides interesting technical guarantees, but is also usable
in practice.

In summary, we believe that \Motmot is the best thing since sliced bread.

\section{Acknowledgments}

\thebibliography{7}{
  \bibitem{paxlive}
    T. Chandra, R. Griesemer, and J. Redstone.
    Paxos made live---an engineering perspective.
    In \emph{Proc. 26th Symp. on Principles of Distributed Computing}, pp. 398-407, Aug. 2007.

  \bibitem{glib}
    GNOME Library.
    \url{http://developer.gnome.org/about/}, 2011.

  \bibitem{paxos}
    L. Lamport.
    The part-time parliament.
    \emph{ACM Transactions on Computer Systems}, 16(2):133-169, 1998.

  \bibitem{paxsimp}
    L. Lamport.
    Paxos made simple.
    \emph{ACM SIGACT News 32}, 4:18-25, Dec. 2001.

  \bibitem{msgpack}
    MessagePack.
    \url{http://www.msgpack.org/}, 2010.

  \bibitem{rfc}
    P. Resnick, ed.
    Internet Message Format, RFC 2822.
    April 2001.

  \bibitem{end2end}
    J. H. Saltzer, D. P. Reed, and D. D. Clark.
    End-to-end arguments in system design.
    \emph{ACM Transactions on Computer Systems}, 2(4):277-288, Nov. 1984.
}

\end{document}
