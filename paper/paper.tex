\documentclass{sig-alternate}

\usepackage{enumitem,url,xspace}
\newcommand\Motmot{\textsc{Motmot}\xspace}
\newcommand\libmotmot{\texttt{libmotmot}\xspace}
\newcommand\libpurple{\texttt{libpurple}\xspace}

\setdescription{leftmargin=0pt}

\begin{document}

\CopyrightYear{2012}

\title{Motmot: A Distributed Chat Protocol}

\numberofauthors{5}
%
\author{
  \alignauthor
  Carl Jackson \\
    \affaddr{Harvard University} \\
    \email{cjackson@college.harvard.edu}
  %
  \alignauthor
  Max Wang \\
    \affaddr{Harvard University} \\
    \email{max.wang@college.harvard.edu}
  %
  \alignauthor
  EJ Bensing \\
    \affaddr{Harvard University} \\
    \email{ebensing@college.harvard.edu}
  %
  \and
  \alignauthor
  Jeff Atwood \\
    \affaddr{Harvard University} \\
    \email{jatwood@college.harvard.edu}
  %
  \alignauthor
  Julie Hansbrough \\
    \affaddr{Harvard University} \\
    \email{jhansbrough@college.harvard.edu}
}

\maketitle

\begin{abstract}

\Motmot is a new distributed text chat protocol based on Paxos. Unlike previous
chat protocols like XMPP, OSCAR, and IRC, \Motmot provides guaranteed message
delivery, total message ordering, end-to-end encryption, peer authentication,
and message deniability.
% TODO: server federates, etc.
% TODO: no xml! It's a feature
% TODO: write more

\end{abstract}

\section{Introduction}

We wrote \Motmot due to our frustrations with existing chat protocols. XMPP, the
protocol underlying Google Chat, is a federated XML-based chat protocol. Most
annoyingly, we found that XMPP did not guarantee message delivery---when one
of the users' connection dies, messages sent around the time of the
disconnection may be lost, without any indication in the protocol that this has
happened. Furthermore, XMPP messages are not totally ordered, and different chat
participants can have different chat logs based on when messages were received.

Finally, XMPP was an unsatisfying protocol technologically. While the protocol
itself is federated, all chat messages are still sent to central servers for
relaying. In addition to the obvious single-point-of-failure concerns, there are
also several product-level concerns. In particular, any server which relays your
messages is able to read the entire contents of the chat. Furthermore, your only
guarantee that you are in fact talking to who you think you are comes from the
servers you are talking to.

Other chat protocols besides XMPP share similar properties. Skype, another
popular chat protocol, acts in a peer-to-peer manner.\footnote{Recent rumors
claim that Skype, under Microsoft's kindly direction, has moved some of the
formerly distributed parts of the Skype protocol to a centrally managed server
cluster. In light of such developments, it's unclear just how peer-to-peer
Skype remains.} However, the protocol (and its primary implementation) are
closed-source, so its guarantees cannot be further examined. As firm believers
in sunlight and open protocols, we found Skype also wanting.

\subsection{Design Goals}

We set out implementing \Motmot with several design goals in mind.

\begin{description}
    \item[Multi-party] Chat sessions can occur between any number of
    individuals.
    \item[Reliability] All chat messages should be reliably delivered. Any message
    recorded in the chat history is guaranteed to be delivered to the rest of the
    \Motmot session.
    \item[Total Ordering] All chat participants will receive the same messages
    in exactly the same order.
    \item[Security] All connections in the system will be encrypted with TLS,
    using certificates (and client certificates) to authenticate the other party
    to the communication.
    \item[Deniability] While participants in the chat are confident that the
    messages they receive are sent by the individuals specified, chat
    participants cannot prove the provenance of any message to any third party.
    In other words, after the chat has completed, all participants can fabricate
    any equivalently valid chat history.
    \item[Federation] Anyone can host their own server and issue authentication
    credentials. Any \Motmot user can talk to any other \Motmot user (with an
    identity in any realm) who correctly implements the protocol.
    \item[Presence] Users can maintain buddy lists, and know the connection
    status (online, busy, idle, etc.) of any buddy.
\end{description}

\section{libmotmot}

Guaranteed delivery and total message ordering are perhaps the most fundamental
chat features provided by \Motmot.  In order to obtain these properties, the
clients participating in a chat must reach a consensus concerning the reception
and order of chat events.  \Motmot assumes a benign failure model and uses a
variant of the Paxos consensus protocol \cite{paxos,paxsimp}, which is the core
of its client library \libmotmot, in order to provide these guarantees.

\subsection{Simplification}

\libmotmot implements a slightly simplified variant of the Multi-Paxos
algorithm, augmented by a number of out-of-band yet nonetheless indispensable
side protocols.  This core Paxos implementation is wrapped by a client
interface and uses MessagePack \cite{msgpack} for network communication.
As with the other components of \Motmot, \libmotmot makes use of the GNOME
Library (GLib) \cite{glib}.

The Paxos implementation of \libmotmot makes a fundamental simplification to
the standard algorithm: we do not allow participants who leave the protocol, by
failure or otherwise, to rejoin.  In the original Paxos protocol, protocol
participants may fail benignly or depart the protocol at any time, and return
at some later point to continue voting and learning values in the system.
This is necessary in order for participants to maintain a notion of persistent
identity.  It is also needed to ensure recovery in the event of total system
failure.  Participants keep a persistent log of the values committed in the
protocol.  In the event of total failure, they can individually come back up,
rejoin the protocol, and corroborate on their individual copies of the log,
thereby allowing the protocol to continue where it left off.

A chat protocol, however, has no need for this mode of recovery, nor for this
particular notion of identity.  The persistent identity which users want from
a chat service can be maintained out-of-band.  If a user leaves a chat, there
is no need, at the level of the consensus protocol, to treat the user as the
\emph{same} user if they returns to the chat later.  Users need not be informed
of the messages that were exchanged in their absence.\footnote{We have not yet
investigated whether or not a distributed chat protocol is able to support
offline messages cleanly.}  So long as the system is able to determine, in a
consistent manner, that the user has left, the user can gain a new identity
within the protocol upon his or her return.

Similarly, if a chat session ends, either due to failure or simply because all
users leave, the termination is allowed to be final.  The participants in the
chat may start a new session if they wish, but this new session need not have
any memory nor knowledge of the old one.

This simplification to Paxos yields many benefits.  For instance, clients may
still choose to keep a record of chats in a local file, but by disallowing
protocol rejoins, this persistent storage does not play any role in the safety
or progress of \libmotmot's Paxos implementation.  We can also avoid the need
for complicated and high-bandwidth recovery mechanisms.  In \libmotmot's Paxos,
as in the final protocol presented in the original Paxos paper, only one
participant is designated as the proposer, and hence allowed to issue prepares
and subsequently decrees, at any given time.  In the absence of rejoins, we
are able to provide a very useful system invariant which full Paxos cannot
have:
\begin{quote}
  If a proposer successfully completes a prepare phase, that proposer will
  remain the protocol's designated proposer until it leaves the protocol, by
  failure or otherwise.
\end{quote}
We discuss this invariant further in the following subsection.

\subsection{Support Protocols}

In order to make Paxos efficient and usable, the core protocol needed to be
supplemented with additional protocols which, despite being out-of-band, were
no less integral to the functionality of \Motmot.  We will describe each of
these side protocols in detail.

\begin{description}
  \item[Join, part, and election:]
    The original Paxos paper remarks on, but does not actually describe, a
    procedure for electing a Paxos system's designated proposer; nor does it
    describe mechanisms for adding and removing individuals from the system.
    These are all, however, necessary to the functionality of our consensus
    algorithm.  Election is necessary in order to ensure that the system makes
    progress past the prepare phase of Paxos, lest multiple participants
    attempt repeatedly to prepare over one another.  Join and part are required
    in order to support any usable notion of chat.

    In \libmotmot, we guarantee the consistency of joins, parts, and election
    results by baking them in to the standard commit protocol.

  \item[Greet protocol:]

  \item[Out-of-band requests:]

  \item[Reconnection:]

  \item[Commit retry:]

  \item[Log synchronization and truncation:]

  \item[Connection continuations:]
\end{description}

\subsection{Implementation}

\subsection{Testing}

\subsection{Lessons Learned}

Our simplification to Paxos is a lesson in the importance of end-to-end design
\cite{end2end}.  The \libmotmot library is a backend protocol layer, on top of
which chat semantics are implemented.  Without taking the semantics of the
system's endpoints---the chat clients---into consideration, our consensus
algorithm would have been unnecessarily featureful and complicated.

The importance of out-of-band effort in distributed protocols was also made
very clear to us in our implementation.  Other implementors of Paxos have also
discovered the substantial amount of support needed to make the the protocol
work in practical production-level settings \cite{paxlive}.  Clean designs
and interfaces for these additional protocols was crucial in keeping our
system simple, functional, and understandable.

\section{Discovery Server}

The discovery server performs several tasks in the system: authentication and
identity federation; presence management; and connection brokering.

\subsection{Identity}

Much like other federated protocols like XMPP and email, identity handles are
based upon DNS. Every domain name with a SRV record can act as a \Motmot
identity realm, and give out email-like addresses (an \verb`addr-spec` as
defined in RFC 2822) for \Motmot communication.

To grant a user an identity on a realm, the user first presents a username and
password to the server, after which the server delegates authority for that name
to the client by signing that client's key. This key-signing process is the
method by which peers can authenticate securely to each other without a
centralized broker.

\subsection{Presence}

In chat protocols, presence is the system by which users can specify a list of
buddies to publish their online status to. The \Motmot discovery server provides
a presence lookup service, by recording the status of all of the clients
connected to it. Servers then push presence updates to all of the other servers
on any user's buddy list, which in turn push the notification to the relevant
clients.

This subsystem system does not need to be robust or reliable, since all pushed
state can be recreated from other sources, and since asynchronously pushed
presence notifications do not need to be perfectly reliable to ensure a usable
user interface.

\subsection{Brokering}

Finally, the server must perform message brokering. For clients behind firewalls
or routers which perform network address translation (NAT), direct connections
are not possible. In such a case, the discovery server might be forced to assist
in connecting the clients.

Technologies like STUN, TURN, and ICE provide implementations of ``NAT poking,''
and while mechanisms for interfacing with such servers were considered in the
design of the server, performing brokering itself was outside the scope of this
version of the protocol.

\subsection{Implementation}

The \Motmot discovery server was implemented in Python using several open source
libraries. The server itself is a custom RPC-like reactor/dispatcher written
using GEvent \cite{glib}, a coroutine-based networking library. For our
communication protocol, we use MessagePack \cite{msgpack} to serialize messages.
This allows for easy cross-language communication with clients and other servers,
since implementations are available for most popular languages.

We found that these two libraries provided good abstractions for the task of
implementing a server. Gevent's coroutines provided an extremely useful
concurrency model: they were executed in a single thread, so synchronization
around global objects was not necessary, while avoiding the complexity (and
unnecessary indentation) of a callback-based system. MessagePack also provided
an excellent abstraction, allowing the server to assume that arbitrary data
structures could be written to and read from the wire. The ad-hoc nature of the
format also helped speed development of the server and protocol.

% TODO: cite pyopenssl
Finally, we used \texttt{openssl} (and the \texttt{pyOpenSSL} library in
particular) to perform certificate operations.

\subsection{Lessons Learned}

% TODO: EJ?

\section{Pidgin Plugin}

To allow our protocol to be utilized via Pidgin, we needed our plugin to
interface with two major aspects of the protocol: the discovery server, and the
\libmotmot library.  A \libpurple protocol plugin is compiled as a statically
linked library and installed into the lib directory of Pidgin. Upon start-up
Pidgin loads each of the protocol plugins in its protocol directory. If the
protocol is successfully loaded, the user can then create accounts of that
specific protocol.  In order to implement a protocol, the programmer must
implement a certain number of functions specifying the code to run when the user
takes an action, for example logging-in or joining a chat, in addition to
listening for input over a socket. Implemented functions are stored in a struct.
Calls to \libmotmot and message passing between the client and the discovery
server had to be implemented within these functions.

\subsection{Plugin Interactions with Discovery Server}
The procedure for hooking up the plugin to the discovery server is similar to
the procedures in other chat protocols. Two user-info fields were added to the
account creation menu, giving the server and port of the discovery server. In
the login code, a connection with the discovery server is initiated, and once
the connection is initiated an input listener is added and authentication
credentials are sent. Once authentication succeeds, status updates and friend
requests are handled. We used message pack in order to have a standard message
format across multiple platforms.

The input listener listens for input from the discovery server. Upon receiving a
message it deserializes the message from the message pack buffer, finds the
opcode, and calls the appropriate \libpurple functions based on the opcode and
the rest of the data sent by the message. Writes to the discovery server are
handled in individual purple functions. Based on the user action, a message pack
object is created, and the data is sent to the server across the connection.
Messages are asynchronous.

\subsection{Plugin Interactions with \libmotmot}
There were a few major difficulties that had to be addressed when calling
\libmotmot within the plugin:

\begin{itemize}
\item \libpurple makes a distinction between ``IMs'' and ``chats''; \libmotmot
does not.  (In particular, IMs are defined as a direct two-person communication
only, whereas chats are an overarching category that can involve one, two, or
more users.) \libpurple takes in different arguments for its IM-related and its
chat-related functions

\item The main Pidgin event loop cannot be modified from within a plugin
(without patching \libpurple itself).  Instead, callbacks are made to plugin
functions from within the main event loop, so we had to find a way to call
\libmotmot functions correctly (i.e. still honoring the correct control flow
from \libmotmot's perspective) and consistently within the bounds of these
plugin functions

\item \libmotmot and \libpurple use different handles to differentiate between
chats and users, and translating between the two handles proved challenging,
especially given the complexity of the information associated with \libpurple
chats (e.g. a single chat may have an id, a room name, \verb`PurpleConnection`
and \verb`PurpleConvChat` objects...)

\end{itemize}

The bulk of the integration work was writing callbacks for the
\verb`motmot_init` function of \libmotmot.  Once these callbacks are written
correctly, \libmotmot handles most of the work of starting, sending, and
coordinating chats automatically.  These callbacks were, in almost all
instances, responsible for calling \libpurple functions, which handle the UI
aspect of the associated \libmotmot calls.  In order to pass relevant
information to our callbacks, we created a \verb`MotmotInfo` struct, a large
data struct that is used differently in different callbacks to get all the
necessary info to the \libpurple calls. \verb`MotmotInfo` contained the info
necessary to mediate between \libpurple and \libmotmot. It contained both the
id of the chat which was necessary for \libpurple and the opaque data handle
necessary for \libmotmot.

We faced a challenge when writing the \verb`enter` function.  This function is a
callback responsible for allowing a user to enter a chat.  Due to the structure
of \libmotmot, the only information passed into \verb`enter` as an argument is
an internal handle, and we need create some sort of external handle.  We solved
this by creating a global variable, \verb`GLOBAL_ACCOUNT`, which represents the
account that a user first logs in under, and we associate our external handle
with this \verb`GLOBAL_ACCOUNT`.  Though this ``works'' if a user only signs in
with on eaccount, this breaks when a user decides to sign in with multiple
accounts at once (something that \libpurple allows), and should be addressed in
future versions.

\section{Conclusions}

Wheeee.

\section{Acknowledgments}

\thebibliography{7}{
  \bibitem{paxlive}
    T. Chandra, R. Griesemer, and J. Redstone.
    Paxos made live---an engineering perspective.
    In \emph{Proc. 26th Symp. on Principles of Distributed Computing}, pp. 398-407, Aug. 2007.

  \bibitem{glib}
    GNOME Library.
    \url{http://developer.gnome.org/about/}, 2011.

  \bibitem{paxos}
    L. Lamport.
    The part-time parliament.
    \emph{ACM Transactions on Computer Systems}, 16(2):133-169, 1998.

  \bibitem{paxsimp}
    L. Lamport.
    Paxos made simple.
    \emph{ACM SIGACT News 32}, 4:18-25, Dec. 2001.

  \bibitem{msgpack}
    MessagePack.
    \url{http://www.msgpack.org/}, 2010.

  \bibitem{rfc}
    P. Resnick, ed.
    Internet Message Format, RFC 2822.
    April 2001.

  \bibitem{end2end}
    J. H. Saltzer, D. P. Reed, and D. D. Clark.
    End-to-end arguments in system design.
    \emph{ACM Transactions on Computer Systems}, 2(4):277-288, Nov. 1984.
}

\end{document}
